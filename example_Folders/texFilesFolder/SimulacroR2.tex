% -----  ejemplo de letra de prueba ----
\documentclass{pruebas}

% para archivos utf8
\usepackage[utf8]{inputenc}
\usepackage[T1]{fontenc}

\usepackage{libertine}
\usepackage{wrapfig}

\usepackage{graphicx}
\usepackage{color}
\usepackage{setspace}
\usepackage{caption}

\setlength\parindent{0pt}

\usepackage{draftwatermark}
\SetWatermarkScale{0.5}
\SetWatermarkText{Example Test}

\graphicspath{{./figs/}}

\usepackage{tikz}
\usepackage{float}

\newcommand*\circled[1]{\tikz[baseline=(char.base)]{
		\node[shape=circle,draw,inner sep=2pt] (char) {#1};}}

\newcommand{\emptybox}[2][\textwidth]{%
	\begingroup
	\setlength{\fboxsep}{-\fboxrule}%
	\noindent\framebox[#1]{\rule{0pt}{#2}}%
	\endgroup
}

\usepackage{fancyhdr}

\pagestyle{fancy}
\fancyhf{}
\lhead{Simulacro Primer Parcial}
\rhead{Página: \thepage}

% ---------------------

\newcommand{\nombres}{Nombre}
\newcommand{\apellidos}{Fuera-de-Lista-4}
\newcommand{\CI}{11111111}
\newcommand{\ejUnovalorA}{30}
\newcommand{\ejUnovalorL}{2}
\newcommand{\ejUnovalorP}{300}
\newcommand{\ejDosvalorI}{52.1 \times 10^{-4}}
\newcommand{\ejDosvalorL}{6}
\newcommand{\ejDosvalorq}{8}
\newcommand{\ejDosvalorP}{38}


\usepackage{tcolorbox}

\newtcolorbox{mybox}[3][]
{
	colframe = #2!80,
	colback  = #2!20,
	coltitle = white,  
	title    = {#3},
	#1,
}


\usepackage{lipsum}


\begin{document}
	
	\begin{minipage}[t]{0.85\textwidth}
		%			\vspace{-1mm}%
		%
		\noindent%
		Unidad Curricular Resistencia de Materiales 2 - Cód. 1312\\
		\textbf{Simulacro Primer Parcial} -- curso 2020\\
		Instituto de Estructuras y Transporte\\
		Facultad de Ingeniería, Universidad de la República\\
	\end{minipage}
	\begin{minipage}[t]{0.1\textwidth}
		\vspace{-5mm}%
		\includegraphics[width=.9\textwidth]{logo_udelar} % .21
	\end{minipage}
	
	\vspace{5mm}
	
	\begin{mybox}{red}{Antes de comenzar a leer, por favor,  \textbf{verifique sus datos personales}}
		%
		\begin{footnotesize}
			\begin{itemize}
				\item Nombre completo: \textbf{ \nombres \, \apellidos}
			\end{itemize}
		\end{footnotesize}
		%
	\end{mybox}
	
%	\vspace{5mm}
%	
%	\hrule
	\vspace{4mm}
	
\ejercicio (6 puntos)

Sea la estructura mostrada en la figura, donde todas las barras tienen sección transversal de área $A=\ejUnovalorA$ cm$^2$ y están formadas por un material de módulo de Young $E=210$ GPa. Además se tiene que $\ell =\ejUnovalorL $ m. %
%
Sobre el nodo $C$ se encuentra aplicada una carga $P=\ejUnovalorP$ kN.

\begin{figure}[htb]
 \begin{center}
   \def\svgwidth{0.7\textwidth}
  \input{./figs/ej1.1.pdf_tex} 
  \end{center}
\end{figure}

Utilizando el Método de las Fuerzas se pide:

\parte Obtener el grado de hiperestaticidad de la estructura
%
\parte Calcular la directa en todas las barras de la estructura en kN.
%
\parte Obtener los valores de las reacciones en los apoyos en kN.
%


		
	
	\newpage
	
%
\ejercicio (4 puntos)

Sea la estructura mostrada en la figura, compuesta por barras formadas por un material de módulo de Young $E$=30 GPa y sección transversal con inercia $I=\ejDosvalorI$ m$^4$. %
%
Las barras $1-2$, $2-3$ y $2-4$ tienen longitud $\ell = \ejDosvalorL$ m y la barra $5-2$ tiene longitud $\ell/2$. El nodo $1$ se encuentra apoyado y los nodos $3$ y $4$ se encuentran empotrados. La barra $1-2$ se encuentra sometida a una carga puntual de valor $P=\ejDosvalorP$ kN en el centro de su vano y la barra $2-4$ se encuentra sometida a una carga uniformemente distribuida de valor $q=\ejDosvalorq$ kN/m. El nodo $5$ se encuentra sometido a una carga puntual de valor $P=\ejDosvalorP$ kN en su extremo libre. 
  
\begin{figure}[H]
	\begin{center}
		 \def\svgwidth{0.7\textwidth}
		\input{./figs/SDestr.pdf_tex} 
	\end{center}
\end{figure}

%	\vspace{2mm}

Considerando el Método de \textit{Slope-Deflection} para la resolución:

\parte Indicar la cantidad mínima de incógnitas para resolver la estructura e indicar a qué  desplazamientos se corresponde cada una de ellas.
%
\parte Enumerar las condiciones a imponer para resolver la estructura.
%
\parte Obtener el/los valor/es numérico/s de las incógnitas cinemáticas del problema indicadas en la parte a).
%
\parte Obtener el valor numérico del momento flector $M_{42}$ según la convención de signos 2 (antihorario positivo). %
%
\parte Realizar el diagrama de momento flector y bosquejar la deformada de la estructura.
 		
	
\end{document}

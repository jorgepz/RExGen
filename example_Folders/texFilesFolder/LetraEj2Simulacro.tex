%
\ejercicio (4 puntos)

Sea la estructura mostrada en la figura, compuesta por barras formadas por un material de módulo de Young $E$=30 GPa y sección transversal con inercia $I=\ejDosvalorI$ m$^4$. %
%
Las barras $1-2$, $2-3$ y $2-4$ tienen longitud $\ell = \ejDosvalorL$ m y la barra $5-2$ tiene longitud $\ell/2$. El nodo $1$ se encuentra apoyado y los nodos $3$ y $4$ se encuentran empotrados. La barra $1-2$ se encuentra sometida a una carga puntual de valor $P=\ejDosvalorP$ kN en el centro de su vano y la barra $2-4$ se encuentra sometida a una carga uniformemente distribuida de valor $q=\ejDosvalorq$ kN/m. El nodo $5$ se encuentra sometido a una carga puntual de valor $P=\ejDosvalorP$ kN en su extremo libre. 
  
\begin{figure}[H]
	\begin{center}
		 \def\svgwidth{0.7\textwidth}
		\input{./figs/SDestr.pdf_tex} 
	\end{center}
\end{figure}

%	\vspace{2mm}

Considerando el Método de \textit{Slope-Deflection} para la resolución:

\parte Indicar la cantidad mínima de incógnitas para resolver la estructura e indicar a qué  desplazamientos se corresponde cada una de ellas.
%
\parte Enumerar las condiciones a imponer para resolver la estructura.
%
\parte Obtener el/los valor/es numérico/s de las incógnitas cinemáticas del problema indicadas en la parte a).
%
\parte Obtener el valor numérico del momento flector $M_{42}$ según la convención de signos 2 (antihorario positivo). %
%
\parte Realizar el diagrama de momento flector y bosquejar la deformada de la estructura.
 